\documentclass[12pt]{article}

% ############################################################# GEOMETRY
\makeatletter% rename layout variables
  \def\layoutwidth{\Gm@layoutwidth}
  \def\layoutheight{\Gm@layoutheight}
  \def\layouthoffset{\Gm@layouthoffset}
  \def\layoutvoffset{\Gm@layoutvoffset}
\makeatother

\usepackage{geometry}
\geometry{
 headsep = 0pt,
 headheight= 0pt, 
  hmarginratio =  1:1,
  vmarginratio = 1:1,
  bindingoffset = 0cm,
  onecolumn,
  a3paper,
  layoutwidth = 210 mm,
  layoutheight = 310 mm,
  layouthoffset=\dimexpr(\paperwidth-\layoutwidth)/2\relax,
  layoutvoffset=\dimexpr(\paperheight-\layoutheight)/2\relax,
  showcrop
}

% ############################################################# ENCODING, FONTS
\usepackage[icelandic, latin, czech]{babel}
\usepackage[utf8]{inputenc}
\usepackage[T1]{fontenc}

\usepackage{textpos}
%\usepackage[showboxes]{textpos}
\usepackage{fix-cm}
% ###################################################### HYPERREF & PDF INFO #

% unicode neccessary so that national characters in hypersetup appear ok
\usepackage[pdftex, unicode, hyperfootnotes=false]{hyperref}

% hyperlinks in black
\makeatletter
\let\Hy@linktoc\Hy@linktoc@none
\makeatother

\hypersetup
  { pdftitle={Pexeso s islandskými slovíčky I.}
  , pdfauthor={Aleš a Dorota Chejnovi}
  , pdfsubject={pexeso}
  , pdfkeywords={pexeso,hvalur.org%
               , islandština, čeština, slovník}
  , bookmarks=false
  , colorlinks=true
  , citecolor=black
  , urlcolor= red-crayola
  , linkcolor=black}

\usepackage{tipa}
\def\dicIPA#1{\textipa{[#1]}}
\usepackage{caption}

\usepackage{graphicx}
\usepackage{tikz}
\usetikzlibrary{calc}
\usepackage{mwe}

% #################################################################### COLORS #

\usepackage{color}
\definecolor {darkgreen}          {rgb} {0.40, 0.01, 0.24}

\definecolor{title}{RGB}{16, 13, 32}
\definecolor{cerulean-dark}{rgb}{0.0, 0.48, 0.65}

\definecolor{alizarin}{rgb}{0.82, 0.1, 0.26}%
\definecolor{applegreen}{rgb}{0.55, 0.71, 0.0}%
\definecolor{blue(pigment)}{rgb}{0.2, 0.2, 0.6}%

\definecolor{burntorange}{rgb}{0.8, 0.33, 0.0}%

\definecolor{byzantium}{rgb}{0.44, 0.16, 0.39}%

\definecolor{darkgoldenrod}{rgb}{0.72, 0.53, 0.04}
\definecolor{darkmagenta}{rgb}{0.55, 0.0, 0.55}
\definecolor{forestgreen(traditional)}{rgb}{0.0, 0.27, 0.13}
\definecolor{ferrarired}{rgb}{1.0, 0.11, 0.0}
\definecolor{goldenpoppy}{rgb}{0.99, 0.76, 0.0}
\definecolor{hanpurple}{rgb}{0.32, 0.09, 0.98}
\definecolor{heliotrope}{rgb}{0.87, 0.45, 1.0}
\definecolor{mediumjunglegreen}{rgb}{0.11, 0.21, 0.18}

% ##### 24 pack Mini Twistables Crayola, 2004
% https://en.wikipedia.org/wiki/List_of_Crayola_crayon_colors?oldformat=true

\definecolor{apricot-crayola}{RGB}{253, 217, 181} %Apricot 	#FDD9B5 
\definecolor{black-crayola}{RGB}{0, 0, 0}%Black 	#000000 
\definecolor{blue-crayola}{RGB}{31, 117, 254}%Blue 	#1F75FE 
\definecolor{blue green-crayola}{RGB}{13, 152, 186}%Blue Green 	#0D98BA 
\definecolor{blue violet-crayola}{RGB}{115, 102, 189}  %Blue Violet 	#7366BD 
\definecolor{brown-crayola}{RGB}{180, 103, 77}%Brown 	#B4674D 
\definecolor{carnation pink-crayola}{RGB}{255, 170, 204}%Carnation Pink 	#FFAACC 
\definecolor{cerulean-crayola}{RGB}{29, 172, 214}%Cerulean 	#1DACD6 
\definecolor{dandelion-crayola}{RGB}{253, 219, 109}%Dandelion 	#FDDB6D 
\definecolor{gray-crayola}{RGB}{149, 145, 140}%Gray 	#95918C 
\definecolor{green-crayola}{RGB}{28, 172, 120} %Green 	#1CAC78 
\definecolor{green yellow-crayola}{RGB}{240, 232, 145}%Green Yellow 	#F0E891 
\definecolor{indigo-crayola}{RGB}{93, 118, 203}%Indigo 	#5D76CB 
\definecolor{orange-crayola}{RGB}{255, 117, 56} %Orange 	#FF7538 
\definecolor{red-crayola}{RGB}{238, 32, 77} %Red 	#EE204D 
\definecolor{red orange-crayola}{RGB}{255, 83, 73}%Red Orange 	#FF5349 
\definecolor{red violet-crayola}{RGB}{192, 68, 143} %	#C0448F 
\definecolor{scarlet-crayola}{RGB}{252, 40, 71}%Scarlet 	#FC2847 
\definecolor{violet purple-crayola}{RGB}{146, 110, 174}%Violet (Purple) 	#926EAE 
\definecolor{violet red-crayola}{RGB}{247, 83, 148}%Violet Red 	#F75394 	
\definecolor{white-crayola}{RGB}{255, 255, 255}%White 	#FFFFFF 
\definecolor{yellow-crayola}{RGB}{252, 232, 131}%Yellow 	#FCE883 	
\definecolor{yellow green-crayola}{RGB}{197, 227, 132}%Yellow Green 	#C5E384 
\definecolor{yellow orange-crayola}{RGB}{255, 174, 66} % 	#FFAE42 

\definecolor{purple mountain majesty-crayola}{RGB}{214,174,221} % 	#D6AEDD
\definecolor{green2-crayola}{RGB}{58,166,85} % 

\usepackage{XCharter}
% #################################################################### LAYOUT #
\usepackage{setspace}
% \normalsize should be {8pt}{9.6pt}
% Two columns layout ruler
\setlength\columnsep    {2\baselineskip}
\setlength\columnseprule{0.4pt}

% Necessary for baseline alignment
\topskip=\baselineskip
\raggedbottom
\setlength\parskip{0pt} % it's better to avoid glue

% Temporarily suppress warnings
\hbadness=5000
\vbadness=5000

\setlength\emergencystretch{17pt}

% Allow smaller emergencystretch in several cases
\newenvironment{xtolerant}[2]{%
  \par
  \ifx\empty#1\empty\else\tolerance=#1\relax\fi
  \ifx\empty#2\empty\else\emergencystretch=#2\relax\fi
}{%
  \par
}
 \renewcommand{\baselinestretch}{0.82}
 
 \graphicspath{%
 % {/home/chejnik/Dokumenty/hvalur.org/images/biolib/full/}%
 % {/home/chejnik/Dokumenty/hvalur.org/images/uploaded_files/}
  {/home/chejnik/Dokumenty/cernobile_kresby_pexeso/}}

\newcommand{\czHyphen}{\rule[.45ex]{.2em}{.11ex}}
\newcommand*{\addthinS}{\hskip0.06667em\relax}
\newcommand*{\addthinSS}{\hskip0.00007em\relax}


% ################################################################ TIKZ COORDINATES #
\def\cropmarkgap{-15}% mm
\makeatletter
\def\Gm@cropmark(#1,#2,#3,#4){% #1 = x direction, #2 = y direction, #3 & #4 no longet used
  \begin{picture}(0,0)
    \setlength\unitlength{1truemm}%
    \linethickness{0.25pt}%
    \put(\the\numexpr #1*\cropmarkgap\relax,0){\line(#1,0){\the\numexpr 20-\cropmarkgap}}%
    \put(0,\the\numexpr #2*\cropmarkgap\relax){\line(0,#2){\the\numexpr 20-\cropmarkgap}}%
  \end{picture}}%
\makeatother

\makeatletter
\def\parsecomma#1,#2\endparsecomma{\def\page@x{#1}\def\page@y{#2}}
\tikzdeclarecoordinatesystem{page}{
    \parsecomma#1\endparsecomma
    \pgfpointanchor{current page}{north east}
    % Save the upper right corner
    \pgf@xc=\pgf@x%
    \pgf@yc=\pgf@y%
    % save the lower left corner
    \pgfpointanchor{current page}{south west}
    \pgf@xb=\pgf@x%
    \pgf@yb=\pgf@y%
    % Transform to the correct placement
    \pgfmathparse{(\pgf@xc-\pgf@xb)/2.*\page@x+(\pgf@xc+\pgf@xb)/2.}
    \expandafter\pgf@x\expandafter=\pgfmathresult pt
    \pgfmathparse{(\pgf@yc-\pgf@yb)/2.*\page@y+(\pgf@yc+\pgf@yb)/2.}
    \expandafter\pgf@y\expandafter=\pgfmathresult pt
}
\makeatother

\newsavebox{\tempbox}
\newsavebox{\tmpbox}
\newsavebox{\tmbox}
\newsavebox{\bbox}
\newsavebox{\bbbox}

\usepackage{eso-pic}
\usepackage{tikzpagenodes}
%\AddToShipoutPicture{\drawbackground}
\newcommand{\shiftleft}{\hspace*{-0.55\dimexpr\csname Gm@layoutwidth\endcsname-\textwidth\relax}}
\newcommand{\shiftup}{\vspace*{-0.13\dimexpr\csname Gm@layoutheight\endcsname-\textwidth\relax}}

% ###################################################### INTERFACE

%DEFINITIONS #
\newcommand\pexesobgcolor{green yellow-crayola}%<== background color
%%% Define Back of all cards
\newcommand\pexesobackcard{karticka_rub3}%<== background color

\newcommand\firstcolor{indigo-crayola} %blue-crayola
\newcommand\secondcolor{indigo-crayola} %blue violet-crayola
\newcommand\thirdcolor{indigo-crayola} %green-crayola
\newcommand\forthcolor{indigo-crayola} %red-crayola



%%% Define phantom
\newcommand\titlebox[1]{%
  \setlength\fboxsep{0pt}% change according to your needs
 % \textcolor{white}{\vphantom{fg}#1}
 \vphantom{fg}#1
}

%%% Define "Array" interface
\makeatletter
    \newcounter{imgs}
    \setcounter{imgs}{0}
    %#1 is the image
    %#2 is the title
    %#3 is the color
    \newcommand{\addimg}[3]{%
        \stepcounter{imgs}%
        \@namedef{imgimage\theimgs}{#1}%
        \@namedef{imgtitle\theimgs}{#2}%
        \@namedef{imgcolor\theimgs}{#3}}
    \newcommand{\getimage}[1]{\expandafter\@nameuse\expandafter{imgimage#1}}%
    \newcommand{\gettitle}[1]{\expandafter\@nameuse\expandafter{imgtitle#1}}%
    \newcommand{\getcolor}[1]{\expandafter\@nameuse\expandafter{imgcolor#1}}%
\makeatother


%%% Define Cards
\addimg{auto}{auto}{\firstcolor}%1
\addimg{autobus}{autobus}{\secondcolor}%
\addimg{babicka}{babička}{\thirdcolor}%
\addimg{kolo}{kolo}{\forthcolor}%
\addimg{boty}{boty}{\firstcolor}%
\addimg{chleba}{chleba}{\secondcolor}%
\addimg{dedecek}{dědeček}{\thirdcolor}%
\addimg{fotak}{foťák}{\forthcolor}%
\addimg{dum}{dům}{\firstcolor}%
\addimg{holcicka}{holka}{\secondcolor}%
\addimg{jablko}{jablko}{\thirdcolor}%
\addimg{kalhoty}{kalhoty}{\forthcolor}%
\addimg{kluk}{kluk}{\firstcolor}%
\addimg{kniha}{kniha}{\secondcolor}%
\addimg{kocka}{kočka}{\thirdcolor}%
\addimg{kvetina}{květina}{\forthcolor}%
\addimg{televize}{televize}{\firstcolor}%
\addimg{les}{les}{\secondcolor}%
\addimg{letadlo}{letadlo}{\thirdcolor}%
\addimg{lod}{loď}{\forthcolor}%
\addimg{majak}{maják}{\firstcolor}%
\addimg{mic}{míč}{\secondcolor}%
\addimg{ledovec}{ledovec}{\thirdcolor}%
\addimg{noviny}{noviny}{\forthcolor}%
\addimg{obrazek}{obrázek}{\firstcolor}%
\addimg{ovce}{ovce}{\secondcolor}%
\addimg{pes}{pes}{\thirdcolor}%
\addimg{pocitac}{počítač}{\forthcolor}%
\addimg{ptak}{pták}{\firstcolor}%
\addimg{ryba}{ryba}{\secondcolor}%
\addimg{skola}{škola}{\thirdcolor}%
\addimg{strom}{strom}{\forthcolor}%32

\addimg{auto}{bíll}{\firstcolor}%
\addimg{autobus}{strætó}{\secondcolor}%
\addimg{babicka}{amma}{\thirdcolor}%
\addimg{kolo}{hjól}{\forthcolor}%
\addimg{boty}{skór}{\firstcolor}%
\addimg{chleba}{brauð}{\secondcolor}%
\addimg{dedecek}{afi}{\thirdcolor}%
\addimg{fotak}{myndavél}{\forthcolor}%
\addimg{dum}{hús}{\firstcolor}%
\addimg{holcicka}{stúlka}{\secondcolor}%
\addimg{jablko}{epli}{\thirdcolor}%
\addimg{kalhoty}{buxur}{\forthcolor}%
\addimg{kluk}{strákur}{\firstcolor}%
\addimg{kniha}{bók}{\secondcolor}%
\addimg{kocka}{köttur}{\thirdcolor}%
\addimg{kvetina}{blóm}{\forthcolor}%
\addimg{televize}{sjónvarp}{\firstcolor}%
\addimg{les}{skógur}{\secondcolor}%
\addimg{letadlo}{flugvél}{\thirdcolor}%
\addimg{lod}{skip}{\forthcolor}%
\addimg{majak}{viti}{\firstcolor}%
\addimg{mic}{bolti}{\secondcolor}%
\addimg{ledovec}{jökull}{\thirdcolor}%
\addimg{noviny}{dagblað}{\forthcolor}%
\addimg{obrazek}{mynd}{\firstcolor}%
\addimg{ovce}{fé}{\secondcolor}%
\addimg{pes}{hundur}{\thirdcolor}%
\addimg{pocitac}{tölva}{\forthcolor}%
\addimg{ptak}{fugl}{\firstcolor}%
\addimg{ryba}{fiskur}{\secondcolor}%
\addimg{skola}{skóli}{\thirdcolor}%
\addimg{strom}{tré}{\forthcolor}%

\addimg{\pexesobackcard}{}{indigo-crayola}%
\addimg{\pexesobackcard}{}{indigo-crayola}%
\addimg{\pexesobackcard}{}{indigo-crayola}%
\addimg{\pexesobackcard}{}{indigo-crayola}%
\addimg{\pexesobackcard}{}{indigo-crayola}%
\addimg{\pexesobackcard}{}{indigo-crayola}%
\addimg{\pexesobackcard}{}{indigo-crayola}%
\addimg{\pexesobackcard}{}{indigo-crayola}%
\addimg{\pexesobackcard}{}{indigo-crayola}%
\addimg{\pexesobackcard}{}{indigo-crayola}%

\addimg{\pexesobackcard}{}{indigo-crayola}%
\addimg{\pexesobackcard}{}{indigo-crayola}%
\addimg{\pexesobackcard}{}{indigo-crayola}%
\addimg{\pexesobackcard}{}{indigo-crayola}%
\addimg{\pexesobackcard}{}{indigo-crayola}%
\addimg{\pexesobackcard}{}{indigo-crayola}%
\addimg{\pexesobackcard}{}{indigo-crayola}%
\addimg{\pexesobackcard}{}{indigo-crayola}%
\addimg{\pexesobackcard}{}{indigo-crayola}%
\addimg{\pexesobackcard}{}{indigo-crayola}%

\addimg{\pexesobackcard}{}{indigo-crayola}%
\addimg{\pexesobackcard}{}{indigo-crayola}%
\addimg{\pexesobackcard}{}{indigo-crayola}%
\addimg{\pexesobackcard}{}{indigo-crayola}%
\addimg{\pexesobackcard}{}{indigo-crayola}%
\addimg{\pexesobackcard}{}{indigo-crayola}%
\addimg{\pexesobackcard}{}{indigo-crayola}%
\addimg{\pexesobackcard}{}{indigo-crayola}%
\addimg{\pexesobackcard}{}{indigo-crayola}%
\addimg{\pexesobackcard}{}{indigo-crayola}%

\addimg{\pexesobackcard}{}{indigo-crayola}%
\addimg{\pexesobackcard}{}{indigo-crayola}%
\addimg{\pexesobackcard}{}{indigo-crayola}%
\addimg{\pexesobackcard}{}{indigo-crayola}%
\addimg{\pexesobackcard}{}{indigo-crayola}%
\addimg{\pexesobackcard}{}{indigo-crayola}%
\addimg{\pexesobackcard}{}{indigo-crayola}%
\addimg{\pexesobackcard}{}{indigo-crayola}%
\addimg{\pexesobackcard}{}{indigo-crayola}%
\addimg{\pexesobackcard}{}{indigo-crayola}%

\addimg{\pexesobackcard}{}{indigo-crayola}%
\addimg{\pexesobackcard}{}{indigo-crayola}%
\addimg{\pexesobackcard}{}{indigo-crayola}%
\addimg{\pexesobackcard}{}{indigo-crayola}%
\addimg{\pexesobackcard}{}{indigo-crayola}%
\addimg{\pexesobackcard}{}{indigo-crayola}%
\addimg{\pexesobackcard}{}{indigo-crayola}%
\addimg{\pexesobackcard}{}{indigo-crayola}%
\addimg{\pexesobackcard}{}{indigo-crayola}%
\addimg{\pexesobackcard}{}{indigo-crayola}%

\addimg{\pexesobackcard}{}{indigo-crayola}%
\addimg{\pexesobackcard}{}{indigo-crayola}%
\addimg{\pexesobackcard}{}{indigo-crayola}%
\addimg{\pexesobackcard}{}{indigo-crayola}%
\addimg{\pexesobackcard}{}{indigo-crayola}%
\addimg{\pexesobackcard}{}{indigo-crayola}%
\addimg{\pexesobackcard}{}{indigo-crayola}%
\addimg{\pexesobackcard}{}{indigo-crayola}%
\addimg{\pexesobackcard}{}{indigo-crayola}%
\addimg{\pexesobackcard}{}{indigo-crayola}%

\addimg{\pexesobackcard}{}{indigo-crayola}%
\addimg{\pexesobackcard}{}{indigo-crayola}%
\addimg{\pexesobackcard}{}{indigo-crayola}%
\addimg{\pexesobackcard}{}{indigo-crayola}%

%%% Global Setup
\newcommand\xspacing{71.3pt}%<== space between the images
\newcommand\yspacing{71.3pt}%<== vertical space between rows
\newcommand\xspacingadd{0.25mm}%<== vertical space between rows
\newcommand\imgperrow{4}%<== number of images per row

\tikzset{p_title/.style={text centered, color=white, minimum height=0.6cm, minimum width=5cm, font=\bfseries}}
\tikzset{p_title_line/.style={ultra thick}}

%%% Define primary for loop
\newcommand{\forloop}[2]{%
    \foreach [count=\i] \x in {#1,...,#2}%<==loop for each image in the array
    {
        \edef\gonode{\noexpand\node[inner sep=0pt] (B) at (A) {\noexpand\includegraphics[width=5cm]{\getimage{\x}}};}%<==Edit to expand the file name
         \gonode%\node [inner sep=0pt] (B) at (A) {\includegraphics[width=5cm]{\getimage{\x}}};%
        \draw [black, thick] ($(B.north west)$) rectangle ($(B.south east)$);%
       \fill[\getcolor{\x}] ($(B.north west)+(0.010,-0.014)$)  rectangle ($(B.north east)+(-0.015,-0.65)$);% 
        \node [p_title] (AA) at ($(B.north west)+(2.5,-0.34)$) {\titlebox{\gettitle{\x}}};%
      %  \draw [p_title_line, color=\getcolor{\x}](AA.south west) -- (AA.south east);%
        \pgfmathparse{Mod(\i,\imgperrow)==0?1:0};%
        \ifnum\pgfmathresult>0
            \coordinate (left) at ([yshift=-\yspacing]left);
            \path let \p1=(left),\p2=(B.south) in coordinate (A) at (\x1,\y2-\yspacing);
        \else
            \coordinate (A) at ([xshift=\xspacing]B.east);%
        \fi
    }}

 %%% Define primary for loop
\newcommand{\forloopback}[2]{%
    \foreach [count=\i] \x in {#1,...,#2}%<==loop for each image in the array
    {
        \edef\gonode{\noexpand\node[inner sep=0pt] (B) at (A) {\noexpand\includegraphics[width=5cm]{\getimage{\x}}};}%<==Edit to expand the file name
         \gonode%\node [inner sep=0pt] (B) at (A) {\includegraphics[width=5cm]{\getimage{\x}}};%
        \draw [black, thick] ($(B.north west)$) rectangle ($(B.south east)$);%
    %   \fill[\getcolor{\x}] ($(B.north west)+(0.010,-0.014)$)  rectangle ($(B.north east)+(-0.015,-0.65)$);% 
     %   \node [p_title] (AA) at ($(B.north west)+(2.5,-0.34)$) {\titlebox{\gettitle{\x}}};%
      %  \draw [p_title_line, color=\getcolor{\x}](AA.south west) -- (AA.south east);%
        \pgfmathparse{Mod(\i,\imgperrow)==0?1:0};%
        \ifnum\pgfmathresult>0
            \coordinate (left) at ([yshift=-\yspacing]left);
            \path let \p1=(left),\p2=(B.south) in coordinate (A) at (\x1,\y2-\yspacing);
        \else
            \coordinate (A) at ([xshift=\xspacing]B.east);%
        \fi
    }}   
    
%%% Define the for loop for 2 images per row on third page
\newcommand{\forlooptwo}[2]{%
    \begingroup
    \def\imgperrow{2}
    \let\originalxspacing\xspacing
    \def\xspacing{5*\originalxspacing}
    \forloop{#1}{#2}\endgroup}
    
\newcommand{\forlooptwoback}[2]{%
    \begingroup
    \def\imgperrow{2}
    \let\originalxspacing\xspacing
    \def\xspacing{5*\originalxspacing}
    \forloopback{#1}{#2}\endgroup}
    %%% x, y coordinates to display the background 
   \newcommand\dpx{0.70}%<== x
    \newcommand\dpy{0.72}%<== y
    
   %  \newcommand\dpx{0.71}%<== x
  %  \newcommand\dpy{0.74}%<== y
    
% ###################################################### DOCUMENT
\begin{document}
\thispagestyle{empty}

% ##################################################### SAVE BOXES
% ####################### half empty back page
\savebox{\bbbox}{%
    \begin{tikzpicture}[remember picture, overlay]
        \coordinate (A) at (page cs:-0.5065,0.595);
        \coordinate (left) at (A);
       \forloopback{65}{68}
        \forlooptwoback{65}{72}
        \forloopback{65}{68}
    \node[inner sep=0pt] (O) at (page cs:0,0)
        {%
\setlength{\fboxsep}{0pt}%
\setlength{\fboxrule}{0pt}%
\fbox{\includegraphics[width=9.99cm]{titleimage3.jpg}}%
};    
    \end{tikzpicture}}
 % ############################ second PAGE
\savebox{\tmbox}{%
  \begin{tikzpicture}[remember picture, overlay]
%\fill[\pexesobgcolor] (page cs:-\dpx,\dpy)  rectangle (page cs:\dpx,-\dpy);
\coordinate (A) at (page cs:-0.5065,0.595);
        \coordinate (left) at (A);
        \forloop{49}{52}
        \forlooptwo{53}{60}
        \forloop{61}{64}
\end{tikzpicture}}
   
% ############################ 3 PAGE    
\savebox{\tempbox}{%
    \begin{tikzpicture}[remember picture, overlay]
        \coordinate (A) at (page cs:-0.5065,0.595);
        \coordinate (left) at (A);
        \forloop{1}{24}
    \end{tikzpicture}}

% ############################ 5 PAGE
\savebox{\tmpbox}{%
\begin{tikzpicture}[remember picture, overlay]
%\fill[\pexesobgcolor] (page cs:-\dpx,\dpy)  rectangle (page cs:\dpx,-\dpy);
       \coordinate (A) at (page cs:-0.5065,0.595);
        \coordinate (left) at (A);
        \forloop{25}{48}
    \end{tikzpicture}}


% ####################### 4 and 6 page
\savebox{\bbox}{%
    \begin{tikzpicture}[remember picture, overlay]
% \fill[\pexesobgcolor] (page cs:-\dpx,\dpy)  rectangle (page cs:\dpx,-\dpy);
        \coordinate (A) at (page cs:-0.5065,0.595);
        \coordinate (left) at (A);
        \forloopback{65}{88}
    \end{tikzpicture}}


    
% ##################################################################### START    
%% first page
\pagestyle{empty}
 \makeatletter
\begin{tikzpicture}[remember picture, overlay]
\fill[\pexesobgcolor] (page cs:-\dpx,\dpy)  rectangle (page cs:\dpx,-\dpy);
 \fill[purple mountain majesty-crayola] (page cs:-0.65,0.65)  rectangle (page cs:0.65,-0.65);
\path (current page.north west) ++(\layouthoffset,-\layoutvoffset)
      ++(0.5\layoutwidth,-0.5\layoutheight) coordinate(Center);
    \node[inner sep=0pt] at (Center) {\usebox{\bbbox}};
\end{tikzpicture}
\makeatother
\begin{textblock}{5}(1.65,0.45)
\begin{center}
   \fontsize{80}{90}\selectfont Pexeso\vspace*{-1ex}
   {\color{red-crayola}{\Huge{\textit{s islandskými slovíčky I.}}}}\\\vspace*{8.7ex}
  \Large{Pojďte se učit islandsky s obrázkovým pexesem!}\\\vspace*{1ex}
   \normalsize{www.hvalur.org}
\end{center}
\end{textblock}
\clearpage

%second page
\thispagestyle{empty}
\makeatletter
\begin{tikzpicture}[remember picture, overlay]
\fill[\pexesobgcolor] (page cs:-\dpx,\dpy)  rectangle (page cs:\dpx,-\dpy);
 \fill[white] (page cs:-0.65,0.65)  rectangle (page cs:0.65,-0.65);  
\path (current page.north west) ++(\layouthoffset,-\layoutvoffset)
      ++(0.5\layoutwidth,-0.5\layoutheight) coordinate(Center);
    \node[inner sep=0pt] at (Center) {\usebox{\tmbox}};
\end{tikzpicture}
\makeatother

\begin{textblock}{5}(1.50,0.35)
   {\Large{Pexeso s islandskými slovíčky I.}}\\\vspace*{2ex}
   Pojďte se učit islandsky s obrázkovým pexesem!\\
   Pexeso obsahuje 64 kartiček. Kartičky si pečlivě vystříhejte. 32 kartiček je s islandskými slovy a 32 kartiček s českými slovy. V tabulce níže se dozvíte, jak slovo vyslovit. Nebojte se; když si nebudete jisti, navštivte \url{www.hvalur.org} a poslechněte si výslovnost.\\\vspace*{1ex}
   \begin{table}[]
\centering
\caption*{Výslovnost islandských slov}
\label{vyslovnost}
\begin{tabular}{llll}
\textbf{} \textit{slovo} & \textit{výslovnost} & \textit{slovo} & \textit{výslovnost} \\ [0.1ex]
\hline \\[-1.5ex]
\textbf{} afi & \dicIPA{{a}{\textlengthmark}{v}{\textsci}} & hús & \dicIPA{{h}{u}{\textlengthmark}{s}} \\
\textbf{} amma & \dicIPA{{a}{m}{\textlengthmark}{a}} & jökull &  \dicIPA{{j}{\oe}{\textlengthmark}{\r{g}}{\textscy}{\textsubring{d}}{\textsubring{l}}} \\
\textbf{} bíll& \dicIPA{{\textsubring{b}}{i}{\textsubring{d}}{\textsubring{l}}} & köttur & \dicIPA{{k\smash{\textsuperscript{h}}}{\oe}{h}{\textsubring{d}}{\textscy}{\textsubring{r}}}  \\
\textbf{} blóm& \dicIPA{{\textsubring{b}}{l}{ou}{\textlengthmark}{\textsubring{m}}} & mynd & \dicIPA{{m}{\textsci}{n}{\textsubring{d}}}  \\
\textbf{} bolti& \dicIPA{{\textsubring{b}}{\textopeno}{\textsubring{l}}{\textsubring{d}}{\textsci}} & myndavél &  \dicIPA{{m}{\textsci}{n}{\textsubring{d}}{a}{v}{j}{\textepsilon}{\textsubring{l}}} \\
\textbf{} bók&  \dicIPA{{\textsubring{b}}{ou}{\textlengthmark}{\r{g}}} & sjónvarp &   \dicIPA{{s}{j}{ou}{n}{v}{a}{\textsubring{r}}{\textsubring{b}}}   \\
\textbf{} brauð& \dicIPA{{\textsubring{b}}{r}{\oe i}{\textlengthmark}{\texttheta}} & skip &   \dicIPA{{s}{\r{\textObardotlessj}}{\textsci}{\textlengthmark}{\textsubring{b}}}  \\
\textbf{} buxur& \dicIPA{{\textsubring{b}}{\textscy}{x}{s}{\textscy}{\textsubring{r}}} & skógur & \dicIPA{{s}{\r{g}}{ou}{\textlengthmark}{\textscy}{\textsubring{r}}}   \\
\textbf{} dagblað& \dicIPA{{\textsubring{d}}{a}{\textbabygamma}{\textsubring{b}}{l}{a}{\texttheta}} & skóli &   \dicIPA{{s}{\r{g}}{ou}{\textlengthmark}{l}{\textsci}}  \\
\textbf{} epli& \dicIPA{{\textepsilon}{h}{\textsubring{b}}{l}{\textsci}} & skór &  \dicIPA{{s}{\r{g}}{ou}{\textlengthmark}{\textsubring{r}}}    \\
\textbf{} fé& \dicIPA{{f}{j}{\textepsilon}{\textlengthmark}} & strákur &   \dicIPA{{s}{\textsubring{d}}{r}{au}{\textlengthmark}{\r{g}}{\textscy}{\textsubring{r}}}  \\
\textbf{} fiskur & \dicIPA{{f}{\textsci}{s}{\r{g}}{\textscy}{\textsubring{r}}} & strætó &  \dicIPA{{s}{\textsubring{d}}{r}{a}{i}{\textlengthmark}{\textsubring{d}}{ou}}   \\
\textbf{} flugvél & \dicIPA{{f}{l}{\textscy}{\textbabygamma}{v}{j}{\textepsilon}{\textsubring{l}}} & stúlka &  \dicIPA{{s}{\textsubring{d}}{u}{\textsubring{l}}{\r{g}}{a}}   \\
\textbf{} fugl & \dicIPA{{f}{\textscy}{\r{g}}{\textsubring{l}}} & tré &  \dicIPA{{t\smash{\textsuperscript{h}}}{r}{j}{\textepsilon}{\textlengthmark}}   \\
\textbf{} hjól & \dicIPA{{\c{c}}{ou}{\textlengthmark}{\textsubring{l}}} & tölva &  \dicIPA{{t\smash{\textsuperscript{h}}}{\oe}{l}{v}{a}}   \\
\textbf{} hundur & \dicIPA{{h}{\textscy}{n}{\textsubring{d}}{\textscy}{\textsubring{r}}} & viti &   \dicIPA{{v}{\textsci}{\textlengthmark}{\textsubring{d}}{\textsci}}  \\
\end{tabular}
\end{table}
\vspace*{2ex}
\noindent\textit{Varianta hry:}\\
{\small{Odkryjte nejdříve kartičku s českým slovem, pak řekněte nahlas islandské slovo a pak teprve odkryjte kartičku s islandským slovem. Jestli jste řekli správné slovo a kartičky mají stejný obrázek, získáváte kartičky k sobě na hromádku.}}\\\vspace*{1ex}
\begin{spacing}{0.7}
\noindent {\scriptsize{\onehalfspacing Autoři: Aleš a Dorota Chejnovi\\
Autorka obrázků: Lucie Peterková\\
Sazba: Aleš Chejn, \LaTeX\\
Tisk: Tiskárna MELMEN, s.r.o., Pardubice\\
Vydal: Aleš Chejn, Husova 1674, Pardubice 53003\\ 
e-mail: \href{mailto:achejn@gmail.com}{achejn@gmail.com}, \url{www.hvalur.org}\\
Náklad: 100 výtisků\\
Rok: 2016}}\\
\end{spacing}
\end{textblock}

%third page
\clearpage
\thispagestyle{empty}
\makeatletter
\begin{tikzpicture}[remember picture, overlay]
\fill[\pexesobgcolor] (page cs:-\dpx,\dpy)  rectangle (page cs:\dpx,-\dpy);
 \fill[purple mountain majesty-crayola] (page cs:-0.65,0.65)  rectangle (page cs:0.65,-0.65);
\path (current page.north west) ++(\layouthoffset,-\layoutvoffset)
      ++(0.5\layoutwidth,-0.5\layoutheight) coordinate(Center);
    \node[inner sep=0pt] at (Center) {\usebox{\tempbox}};
\end{tikzpicture}
\makeatother

% forth page
\clearpage
\thispagestyle{empty}
\makeatletter
\begin{tikzpicture}[remember picture, overlay]
\fill[\pexesobgcolor] (page cs:-\dpx,\dpy)  rectangle (page cs:\dpx,-\dpy);
 \fill[purple mountain majesty-crayola] (page cs:-0.65,0.65)  rectangle (page cs:0.65,-0.65);
\path (current page.north west) ++(\layouthoffset,-\layoutvoffset)
      ++(0.5\layoutwidth,-0.5\layoutheight) coordinate(Center);
    \node[inner sep=0pt] at (Center) {\usebox{\bbox}};
\end{tikzpicture}
\makeatother

%fifth page
\clearpage
\thispagestyle{empty}
\makeatletter
\begin{tikzpicture}[remember picture, overlay]
\fill[\pexesobgcolor] (page cs:-\dpx,\dpy)  rectangle (page cs:\dpx,-\dpy);
 \fill[purple mountain majesty-crayola] (page cs:-0.65,0.65)  rectangle (page cs:0.65,-0.65);
\path (current page.north west) ++(\layouthoffset,-\layoutvoffset)
      ++(0.5\layoutwidth,-0.5\layoutheight) coordinate(Center);
    \node[inner sep=0pt] at (Center) {\usebox{\tmpbox}};
\end{tikzpicture}
\makeatother

%sixth page
\clearpage
\thispagestyle{empty}
\makeatletter
\begin{tikzpicture}[remember picture, overlay]
\fill[\pexesobgcolor] (page cs:-\dpx,\dpy)  rectangle (page cs:\dpx,-\dpy);
 \fill[purple mountain majesty-crayola] (page cs:-0.65,0.65)  rectangle (page cs:0.65,-0.65);
\path (current page.north west) ++(\layouthoffset,-\layoutvoffset)
      ++(0.5\layoutwidth,-0.5\layoutheight) coordinate(Center);
    \node[inner sep=0pt] at (Center) {\usebox{\bbox}};
\end{tikzpicture}
\makeatother

\clearpage
\end{document}
